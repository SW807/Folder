\section{Opkald}
Via den indbyggede opkaldsoversigt i Android er det muligt at se på en patients opkalds historik. Dette kommer med begrænsningen at applikationer skal have tilladelse før de kan bruge den.

\subsection{Antal taget/ikke taget opkald}
Da der er en risiko for at folk der er deprimerede eller på vej ind i en depression lukker sig af for omverdenen, er vores tanke at man ud fra antallet af opkald en patient vælger at besvare og mere vigtigt vælger ikke at besvare, kan give en indikation af patientens mentale tilstand. På samme tid kunne man også få en indikation på deres sociale liv ved at se hvor mange opkald de overhovedet lavet.
Dette kan være et unøjagtigt måleredskab, da folk sagtens kunne vælge at tage deres telefon for at skjule hvor dårligt de egentlig har det, men for dem der ikke gør det kan det være en mulighed for at registrere depression før det bliver alvorligt.

\subsection{Varighed}
Varigheden af telefonsamtaler kan bruges til at se ændringer i en persons mentale tilstand, hvis fx en person der normalt snakker mindst 10 minutter når folk ringer til ham ikke har haft telefonsamtaler på over 2 minutters varighed i den sidste uge, er det muligt der kunne være noget galt.

\begin{comment}
Idéer til Opkald:
\begin{description}[style=nextline]
\item[Antal taget/ikke taget]
\item[Varighed]
\item[Andre?]
\end{description}

Sensor
\begin{description}[style=nextline]
\item[Kan vi få fat i data?] Ja, der er en CallLog i Android som kan bruges til dette, hvorfra man kan få data om de sidste \~500 calls.
\item[Er der begrænsninger?] Man skal bruge permission. Der kan være regler vi skal tænke på med hensyn til persondataloven, men hvis vi aggregerer dataen og ikke gemmer individuelle calls kan vi nok undgå den problematik. 
\item[Hvilke data giver sensoren?] Varighed, til og fra, om opkaldet blev taget eller ej, tidspunkt, sted(måske), type af kald(Indgående, udgående, misset) osv.
\end{description}

Analyse
\begin{description}[style=nextline]
\item[Har vi data nok?] Ja, den er begrænset til 500 opkald men vi kan gemme den relevante data i en lokal database og gør det ubegrænset(Persondataloven?)
\item[Hvilke data skal benyttes?] Alle de data vi kan få fra loggen som giver mening at registrere, f.eks. varighed eller type.
\item[Formål med analysen] Det primære formål med denne data er at se på social aktivitet, f.eks. hvor ofte og hvor længe brugeren snakker i telefon, eller om de f.eks. undgår opkald.
\item[Ide til visualisering (?)] Kalender?
\item[(kort oprids af fremgangsmåde)] N/A
\item[(gem illustrationer og kilder)] http://developer.android.com/reference/android/provider/CallLog.html http://developer.android.com/reference/android/provider/CallLog.Calls.html
\end{description}
\end{comment}