\section{Lyssensor}

\subsection{Sensor}

\begin{description}
\item[Kan vi få fat i data?]
Ja.

\item[Hvilke data giver sensoren?]
Giver en enkelt talværdi i lux.

\item[Er der begrænsninger?]
Få enheder har en lys-sensor.

\end{description}

\subsection{Analyse}
\begin{description}
\item[Har vi data nok?]
Hvis sensoren er til stede i enheden, bør det være muligt at skelne mellem tændt og slukket lys.
Afhængigt af kvaliteten af sensoren, kan der skelnes mellem forskellige lysniveauer.

\item[Hvilken data skal benyttes?]
Talværdi i lux, der repræsenterer lysniveau.

\item[Hvad er formål med analysen?]
At afgøre om en person opholder sig meget i mørke eller ikke.

Evt. til at afgøre om mobilen ligger i en lomme eller ligger fremme, så de andre logninger er relevante eller ej.

\item[Hvordan kan det visualiseres?]

\end{description}

\subsection{Teknisk}
\url{http://developer.android.com/guide/topics/sensors/sensors_environment.html}

SensorManager bruges til at anmode om sensor til en Sensor.TYPE\_LIGHT, hvor der kan registreres en listener.
