\section{Tekstanalyse}
\subsection{Sensor}
	\begin{description}[style=nextline]
		\item[Kan vi få fat i data?]
		På Android kan man få adgang til SMS beskeder. Hvor tankegangen er at disse analyseres til at finde ud af en persons sindsstemning.
		\item[Er der begrænsninger]
		Hvis man i manifestet skal have rettighed til at læse SMS'er er der ikke begrænsninger.
		Dog er der spørgsmålet om privatliv der skal overvejes.
		\item[Hvilke data gives der?]
		Sms beskeder, d.v.s. en mængde af ustrukturerede tekster med tilknyttet metadata, såsom afsendelsestidspunkt og modtager.
	\end{description}
	
\subsection{Analyse}
	\begin{description}[style=nextline]
		\item[Har vi data nok?]
		Hvis man skal lave sentiment analysis kræver det en stor mængde af træningsdata, for at tekstanalysen bliver akkurat. Twitter kunne evt. være en ressource til at finde sådanne data. Problemet er at det skal være på dansk, og er nok her en stor del af arbejdet med sentiment analysis af danske SMS tekster ligger.
		\item[Hvilke data skal benyttes?]
		Selve SMS teksterne vil være centrale i tekstanalyse til at blive analyseret på.
		Dog kan andre metadata såsom afsendelsestidspunkt, modtager, længde, og respons måske også være interessante at kigge på
		
		Afsendelsestidpunkt kunne e.v.t. være interessant til at se om sindstilstand er kontinuert, således at man kan se om en negativ ladet sms tekst er vedvarende over en længere periode eller om det er et enkeltstående tilfælde.
		
		Modtager kunne være interessant til at se om man til personer man normalt har en ensformig tekst stemning mod ændrer sig til eksempelvis meget negativt ladede tekster.
		
		Længden af SMSerne kan eksempelvis bruges til at registere adfærdsændringer, eksempelvis en person der typisk har SMS'er med en længde på 10 ord i gennemsnit pludselig får et gennemsnit på 30+ ord.
		
		Respons kan bruges til at begrunde hvorfor man har et givent toneleje i sine SMS'er. baseret på tonelejet af de SMS'er der svares på.
		\item[Formål med analysen]
		At kunne få en indikation på en persons sindstilstand.
		\item[Idé til visualisering]
		For et øjebliksbillede kunne man give et 
	\end{description}