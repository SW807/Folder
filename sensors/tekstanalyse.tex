\section{Tekstanalyse}
På Android kan man få adgang til SMS beskeder. Hvor tankegangen er at disse analyseres til at finde ud af en persons sindsstemning. Det kræver rettighed til applikationen til at læse SMS'er, man skal også overveje brugerens privatliv. Data er SMS beskeder, d.v.s. en mængde af ustrukturerede tekster med tilknyttet metadata, såsom afsendelsestidspunkt og modtager.

\subsection{Antal, keywords og andre metadata}
Man kan hive ud data som antal beskeder brugeren modtager, og om der er nogle specifikke keywords som kunne være interessante fra en psykiatrisk(F.eks. hvis de bruger ordet "selvmord" ofte). Andre metadata som afsendelsestidspunkt, modtager, respons kan også være interessante at kigge på og disse kan også nemt udtrækkes \citep{misc:androidsmsread}.

Afsendelsestidspunkt kunne e.v.t. være interessant til at se om sindstilstand er kontinuert, således at man kan se om en negativ ladet sms tekst er vedvarende over en længere periode eller om det er et enkeltstående tilfælde.
		
Modtager kunne være interessant til at se om man til personer man normalt har en ensformig tekst stemning mod ændrer sig til eksempelvis meget negativt ladede tekster.

Længden af SMSerne kan eksempelvis bruges til at registere adfærdsændringer, eksempelvis en person der typisk har SMS'er med en længde på 10 ord i gennemsnit pludselig får et gennemsnit på 30+ ord.

Respons kan bruges til at begrunde hvorfor man har et givent toneleje i sine SMS'er. baseret på tonelejet af de SMS'er der svares på.

Denne slags analyse kan give indsigt ind i brugerens sindstilstand. 

\subsection{Sentiment analysis}
Hvis man skal lave sentiment analysis kræver det en stor mængde af træningsdata, for at tekstanalysen bliver akkurat. Twitter kunne evt. være en ressource til at finde sådanne data. Problemet er at det skal være på dansk, og er nok her en stor del af arbejdet med sentiment analysis af danske SMS tekster ligger.
Det vil være selve SMS teksterne der er centrale i tekstanalyse når det kommer til sentiment analysis. 
Hvis denne kun gøres præcis kunne man få en meget god indikation på en brugers sindtilstand.

Til dette skulle man bruge træningsdata som kan bruges til at opbygge en sentiment analysis classifier, eksempelvis Naïve Bayes Classifier. Denne classifier vil så kunne bruges til at vurdere sindstilstand for brugere. Hvis træningsdata ikke er tilgængeligt kunne man se på negative eller positive ord og hvordan de bruges i SMS teksen.

\subsection{Visualisering}
Man kunne bruge disse til at få et øjebliksbillede som kunne vises til brugeren eller brugerens psykiater/psykolog i form af en beskrivende besked om sindstilstand, evt. med tal for hvor sikker vi er på denne bedømmelse. Her forstilles det en skala over sindstilstande, hvor forskellige nivauer haren dertilhørende beskrivelse.
Hvis man skulle tage analysen i et kronologisk perspektiv kunne en graf med tid og sindstilstandsniveau være en mulighed. 

\begin{figure}
	\centering
	\begin{subfigure}[b]{0.4\textwidth}
		\includegraphics[width=\textwidth]{positiveSMS}
		\caption{Positiv SMS}
	\end{subfigure}
	~
	\begin{subfigure}[b]{0.4\textwidth}
		\includegraphics[width=\textwidth]{negativeSMS}
		\caption{Negativ SMS}
	\end{subfigure}
	\caption{Eksempler på SMS'er}
\end{figure}
	
	