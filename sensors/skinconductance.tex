\section{Hudledeevne}
Idéer til galvanisk hud respons:
\begin{description}[style=nextline]
\item[Stress måling baseret på galvanisk hud respons(sved)]
\item[Motion]
\end{description}

Sensor
\begin{description}[style=nextline]
\item[Kan vi få fat i data?] Dataen er tilgængelig ved brug af JawBone.
\item[Er der begrænsninger?] Det kræver at man har købt en JawBone først. Nogle psykiske sygdomme, f.eks psykopati ville dette åbenbart ikke være en effektiv måling af stress. Det kræver at personen bruger JawBone hele tiden.
\item[Hvilke data giver sensoren?] Sensoren giver data omkring hvor godt ens hud leder strøm, eller hvor svedig brugeren nu engang er. Sensoren kan også give andre data, men dette er uvigtigt for dette emne.
\end{description}

Analyse
\begin{description}[style=nextline]
\item[Har vi data nok?] Det er nok ikke et problem at skaffe data nok.
\item[Hvilke data skal benyttes?] Sved måling.
\item[Formål med analysen] Når ens galvanisk hud respons er højere end en baseline niveau kan man enten antage at personen er stresset eller motionerer. Hvis tallet er højere i en længere periode vil det med stor chance betyde at vedkommende er stresset, da de fleste personer kun motionerer i 30-60 minutter.
\item[Ide til visualisering (?)] Graf over dagens sved niveau.
\item[(kort oprids af fremgangsmåde)] Få et baseline niveau hvor brugeren ikke er stresset eller motionerer, og baseline hvor personen er stresset og motionerer. Dette kan man så bruge til at gætte om personen motionerer eller er stresset. Hvorefter man så kan lave en graf, og annoterer de tidspunkter hvor systemet tror at brugeren har været stresset eller har motioneret.
\item[(gem illustrationer og kilder)] JawBone UP3: https://jawbone.com/blog/up3-wearable-heart-rate-monitor/
\end{description}