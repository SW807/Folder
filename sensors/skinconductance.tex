\section{Galvanisk Hud Respons}
For at benytte galvanisk hud respons kræver det at man har en JawBone.
En JawBone giver adgang til data omkring hvor god ens hud er til at lede strøm, huden leder strøm bedre jo mere man sveder, og derfor giver det også data omkring hvor meget man sveder.


\subsection{Stress måling}
Ens galvanisk hud respons gennemsnit over en dag kan give en ide om en person generalt sveder meget i løbet af en dag, hvilket muligvis man indikere at man er stresset.
Dette vil derfor kunne indikere om man er på vej ind i en depression pga. stress.
Denne form for måling vil derfor være en hjælp til brugeren ved at give besked om at man måske skal lave noget afstressende og på den måde undgå at man kommer i en depression.

\subsection{Motion}
Motion er en vigtig for alle personer, men det kan dog være ekstra vigtigt hvis man lider af en psykisk sygdom som depression eller maniodepression.
Med denne type målinger vil man kunne se om en person dyrker meget eller lidt motion, dette kan gøres ved at se på de galvaniske hud respons tal, og se om de pludseligt ændre sig meget i en kort periode, hvis den gør det ville dette kunne indikere at man dyrker motion.
Derfor ville denne form for måling kunne benyttes til advarer brugeren om at vedkommende skal til at dyrke lidt motion, men også hvis man er vant til at dyrke motion og man pludselig stopper med det, så kan dette muligvis indikere at vedkommende er på vej ind i en negativ periode.

\begin{comment}

Idéer til galvanisk hud respons:
\begin{description}[style=nextline]
\item[Stress måling baseret på galvanisk hud respons(sved)]
\item[Motion]
\end{description}

Sensor
\begin{description}[style=nextline]
\item[Kan vi få fat i data?] Dataen er tilgængelig ved brug af JawBone.
\item[Er der begrænsninger?] Det kræver at man har købt en JawBone først. Nogle psykiske sygdomme, f.eks psykopati ville dette åbenbart ikke være en effektiv måling af stress. Det kræver at personen bruger JawBone hele tiden.
\item[Hvilke data giver sensoren?] Sensoren giver data omkring hvor godt ens hud leder strøm, eller hvor svedig brugeren nu engang er. Sensoren kan også give andre data, men dette er uvigtigt for dette emne.
\end{description}

Analyse
\begin{description}[style=nextline]
\item[Har vi data nok?] Det er nok ikke et problem at skaffe data nok.
\item[Hvilke data skal benyttes?] Sved måling.
\item[Formål med analysen] Når ens galvanisk hud respons er højere end en baseline niveau kan man enten antage at personen er stresset eller motionerer. Hvis tallet er højere i en længere periode vil det med stor chance betyde at vedkommende er stresset, da de fleste personer kun motionerer i 30-60 minutter.
\item[Ide til visualisering (?)] Graf over dagens sved niveau.
\item[(kort oprids af fremgangsmåde)] Få et baseline niveau hvor brugeren ikke er stresset eller motionerer, og baseline hvor personen er stresset og motionerer. Dette kan man så bruge til at gætte om personen motionerer eller er stresset. Hvorefter man så kan lave en graf, og annoterer de tidspunkter hvor systemet tror at brugeren har været stresset eller har motioneret.
\item[(gem illustrationer og kilder)] JawBone UP3: https://jawbone.com/blog/up3-wearable-heart-rate-monitor/
\end{description}
\end{comment}