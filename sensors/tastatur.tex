\section{Tastatur}
Det er umiddelbart ikke muligt at få adgang til information fra tastaturet, så hvis vi vil have data om tastaturets brug skal man lave sit eget tastatur fra bunden eller have specielle tilladelser.

Igennem tastaturet kunne det måske have været muligt at se hvor usikre folk var, da vi fx kunne se på hvor mange gange de skiftede mening angående konstruktion af beskeder, altså hvis de sletter deres ord eller starter forfra meget.
Desuden kunne man måske evaluere på hvad patienters humør var baseret på hvor hårdt de trykker på tastaturet.

\begin{comment}
Idéer til tastatur:
\begin{description}[style=nextline]
\item[Antal Rettelser]
\item[Ordbrug]
\item[Tid for en sætning]
\item[Tryk niveau]
\end{description}

Sensor
\begin{description}[style=nextline]
\item[Kan vi få fat i data?] Lyder som om det ikke er muligt at keylogge globalt. Man kan lave sit eget tastatur som erstatter default tastatur og keylogge derfra, men det lyder ikke til at det er muligt at lave en Service keylogger som får fat i alt som bliver skrevet på devicet.
\item[Er der begrænsninger?] Det lyder til at man bliver nødt til at lave sit eget tastatur.
\item[Hvilke data giver sensoren?] N/A
\end{description}

Da vi ikke tror det er praktisk kan denne nok blive udelukket som en mulighed og vi vil derfor ikke forsætte med analysen.

Analyse
\begin{description}[style=nextline]
\item[Har vi data nok?]
\item[Hvilke data skal benyttes?]
\item[Formål med analysen]
\item[Ide til visualisering (?)]
\item[(kort oprids af fremgangsmåde)]
\item[(gem illustrationer og kilder)]
\end{description}
\end{comment}




