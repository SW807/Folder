\section{Lyd}
Der er mikrofoner i alle smartphones, der muliggør indsamling af lyd data.
Dette kan også sættes op til at optage mens en telefonsamtale er i gang, så brugeren der har app'en installeret får sin stemme optaget.
Derudover er der den enkle løsning at patienten optager sig selv.

En begrænsningen ved at optage samtaler er at det kun er patientens stemme man kan optage, og således ikke den han snakker med.

Det optagne lyd gemmes som en fil der kan bruges til efterfølgende analyse af bl.a. amplitude og pitch.

\subsection{Parkinson}
Analysen af patientens stemme kan bruges til at detektere Parkinsons\citep{6168572}.
Algoritmen der findes i kilden udnytter at personer med Parkinson syge er ringere til at snakke.

Problematikken ligger i at skaffe data fra en mængde af patienter med og uden psykiske lidelser, for at få en præcis klassificer til brug af algoritmerne beskrevet i \citep{6168572,6346375}.

\subsection{Maniodepression diagnosticering}
En anden mulighed er at forske nærmere i at bruge stemmeanalyse til at diagnosticere sygdomme såsom maniodepression \citep{6346375}, eller andre sygdomme der har effekt på stemmen.

Derudover kan mani muligvis detekteres ved at måle på amplituden og stemmelejet, men kræver mere forskning på området \citep{6346375}.
Dog har de haft gode resultater med individuelle patienter, men det viser sig at stemmelejet ikke kan bruges generelt set til detektering af mani, det er noget man skal vurdere fra patient til patient.
	
\subsection{Idé til visualisering}
Man kunne eksempelvis lave en graf med stemmelejet eller amplitude over tid, for at se om ens stemme udvikler sig på usædvanlig vis (snakker højt/hurtigt).
Et alternativ er et simpelt binært resultat til at afgøre om man har en given sygdom eller ej, evt. med et dertilhørende konfidensniveau.
