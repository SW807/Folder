\section{Lokation}
Kilde: \url{https://developer.android.com/training/location/index.html}

\subsection{Sensor}

\begin{description}
\item[Kan vi få fat i data?]
Ja, der kan anmodes om lokation på et væld af måder; seneste lokation, anmode enkelt opdatering, anmode flere opdateringer.
Dette afhænger af rettigheder til brug af henholdsvis grov og/eller fin lokation.
\item[Hvilke data giver sensoren?]
Breddegrad, længdegrad, præcision i meter, højde i meter, retning i grader, dato-tidsstempel, hastighed i m/s, leverandør (GPS el. netværk).
\item[Er der begrænsninger?]
Opdateringer afhænger af andre apps' anmodninger, samt forbindelse(r).
Det er muligt at sætte minimum og maksimum tid, men kan ikke garanteres at minimum overholdes.
Hyppige opdateringer er strømkrævende.
\end{description}

\subsection{Analyse}
\begin{description}
\item[Har vi data nok?]
Sikkert.
\item[Hvilken data skal benyttes?]
Breddegrad, længdegrad, præcision og datotidsstempel.
Højde og hastighed er mindre anvendeligt, og er mere usikre (både ift. præcision og om der overhovedet er en værdi).
\item[Hvad er formål med analysen?]
Hvor meget man bevæger sig; hvor meget er man udenfor hjem/arbejde? Hvor meget opholder man sig hjemme? Hvor er man oftest (nøgle-lokationer, fx. venner, familie, værtshus)?
\item[Hvordan kan det visualiseres?]

\end{description}

\subsection{Teknisk}
Google Play Services (foretrukket) eller Android framework location APIs.
\url{http://developer.android.com/reference/android/location/LocationManager.html}
