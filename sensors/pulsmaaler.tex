\section{Pulsmåler}
For at kunne benytte sig af pulsmåleren, kræver det at brugeren har en JawBone eller anden hardware som kan måle puls. Dog er der også en metode til at måle puls med bare en mobil hvor man sætter fingeren på kameraet når lyset er tændt, men dette har tvivlsom kvalitet.

\subsection{Aktivitet}
Pulsmåleren kan bruges til at indikere om brugeren er i gang med at dyrke motion, da når man dyrker motion stiger ens puls i en periode.
Eftersom motion hjælper alle personer med at have et positivt indblik på ting, kan det derfor være vigtigt for nogle personer at opretholde en aktiv hverdag.
Pulsmåleren kan også være behjælpelig for brugeren så han kan se om han dyrker nok motion i forhold til det han er blevet anbefalet, hvis dette kan være tilfældet.

\subsection{Anfald}
En pulsmåler gør det også muligt at registrere om patienter får visse typer af anfald, fx. angst eller panik, hvis man er i stand til at registrere hvornår disse sker kan man måske nemmere nå frem til hvad der forårsagede det og give kontekst til patienters læger om hvad der er sket siden sidst.

\begin{comment}
Idéer til pulsmåler:
\begin{description}[style=nextline]
\item[Puls]
\item[Aktivitet]
\end{description}

Sensor
\begin{description}[style=nextline]
\item[Kan vi få fat i data?] Der er måder at målepuls baseret på kun det hardware der er i en Android telefon men baseret på erfaring med dem er de ikke særlig akkurate. Der skal nok ekstra hardware til der kan rapportere det akkurat.
\item[Er der begrænsninger?] Ja, telefonen kan ikke gøre det godt nok af sig selv.
\item[Hvilke data giver sensoren?] Puls.
\end{description}

Analyse
\begin{description}[style=nextline]
\item[Har vi data nok?] Hvis der er hardware der kan rapportere det regulært tror vi der er data nok.
\item[Hvilke data skal benyttes?] Puls.
\item[Formål med analysen] Måling af puls kan bruges til at estimere aktivitetsniveau, men på samme tid kan det også bruges til at f.eks. måle om man er stresset/bange/spændt. Det kan også bare bruges til at finde ud af hvad din hvilende puls er, som igen giver indtryk til aktivitetsniveau.
\item[Ide til visualisering (?)] Vis en graf over dagene der viser dit puls niveau. Hvis man har en generel meget høj puls kan man nok sige at brugeren har man nok dårlig kondition, og hvis de har en generel meget lav puls er man nok i god kondition. Man kan også visualisere motionering hvor man har en periode af højere aktivitet end gennemsnits niveauet. Man kan også visualisere om man sover da pulsen kan falde kraftigt, og se om de sover godt(pga stabil lav puls) eller om de sover dårligt(ustabil normal puls).
\item[(kort oprids af fremgangsmåde)] Få en baseline niveau af brugerens pulsniveau, og derefter brug dette tal til at se om man har en stresset periode. Dette kræver nok at brugeren ikke er stresset når man får baseline.
\item[(gem illustrationer og kilder)] https://jawbone.com/ Se JawBone UP3, UP2 serien læser ikke puls.
\end{description}
\end{comment}